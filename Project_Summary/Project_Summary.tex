\documentclass[11pt,letterpaper,english]{article}
\usepackage[T1]{fontenc} % Standard package for selecting font encodingsamely titles, headers and contents
\usepackage{txfonts} % makes spacing between characters space correctly
\usepackage{xcolor} % Driver-independent color extensions for LaTeX and pdfLaTeX.
%\usepackage[pagestyles,raggedright]{titlesec} % related with sections—n
%\usepackage{blindtext} % To create text
\usepackage{fancyhdr} % header footer placement

\usepackage[top=1in, bottom=1in, left=1in, right=1in] {geometry} % Margins
\usepackage{graphicx}  % Essential for adding images to you document.

\usepackage{sectsty}
\sectionfont{\normalsize}
\subsectionfont{\normalsize}
\subsubsectionfont{\normalsize \it}

\usepackage{caption}
\captionsetup{labelsep=period}


\pagestyle{fancy} % allows you to use the header and footer commands

\raggedright
\begin{document}

\setlength{\parindent}{0in} % Amount of indentation at the first line of a paragraph.

\pagestyle{fancy} \lhead{The Detailed Physical Structure of the Circumgalactic Medium} \rhead{Evan Schneider} \renewcommand{%
\headrulewidth}{0.0pt}

\begin{center}
\bf {PROJECT EXECUTIVE SUMMARY} \\
{\bf \small {\em (Must not exceed 1 page but does} not {\em count toward the 15-page project narrative limit.)}}
\end{center}


%\begin{flushleft}
\bigskip

\textbf{Title (\emph{80 characters max; strictly enforced})}: The Detailed Physical Structure of the Circumgalactic Medium \smallskip

\textbf{PI and Co-PI(s)}: Evan Schneider (PI, Princeton University), Drummond Fielding (co-PI, Flatiron Institute) \smallskip

\textbf{Applying Institution/Organization}: Princeton University \smallskip

\textbf{Number of Processor Hours Requested}: 530k node-hours (Summit) \smallskip

\textbf{Amount of Storage Requested}: 500 TB \smallskip

\textbf{Executive Summary ({\emph{May use the remainder of page}}):} \\

%The executive summary should accurately describe the proposed research and the high-impact scientific or technical advances you will realize with the proposed INCITE allocation. Industry organizations should also summarize the potential economic or strategic business impact of the proposed research.\\
\vspace{.15in}

%Why is the CGM important? Why don't we understand it?
Three of the foremost questions in modern astrophysics are (i) what fuels the growth of galaxies, (ii) what quenches the growth of some galaxies, and (iii) where does most of the matter in universe reside. The answers to all of these questions depend on the cycling of gas through the circumgalactic medium (CGM) -- the dilute gas that surrounds galaxies and fills their dark matter halos, and regulates the growth of galaxies over cosmic time. Understanding the physical state of the CGM is therefore critical if we are to develop a coherent theory of galaxy growth and evolution. Until this point, however, a detailed understanding of the CGM has been challenging to obtain.
\vspace{.15in}

%What is our project?
The halos of galaxies contain hot, pressure-supported, turbulent flows that are subject to thermal instability. The development and evolution of a turbulent multiphase plasma is a non-linear process that requires three-dimensional numerical simulations to capture fully. Gas within the CGM exists at a range of pressures, and turbulence may be stirred with subsonic or supersonic velocities in different regions. Resolving the range of scales involved, from the driving scale of the turbulence (of order the scale of the galaxy halo, $R\sim 100$ kpc), to the length scale of the coolest clouds that may reside within it ($\ell_{\rm cool} << 1$ kpc) is a daunting numerical challenge.
\vspace{.15in}

%How will we do our project?
In recent years, a vast amount of effort by the scientific community has been funneled into creating both hardware and software that can be used to address the most demanding computational challenges. Results of these efforts include new leadership class facilities, such as the GPU-powered IBM/NVIDIA system "Summit", as well as codes that can leverage the power of these machines. With our GPU-native hydrodynamics code {\tt Cholla} (Computational Hydrodynamics On paraLLel Architectures) and the Summit system, we have for the first time the ability to model the detailed multiphase structure of the CGM across the range of necessary scales and environments in a fully-resolved way.
\vspace{.15in}

%What will the impact be on the community?
The results of these simulations will have a far-reaching impact. As the first simulations capable of resolving the full range of scales at play in the CGM, this project will immediately have the potential to explain the puzzling observations of multiphase gas around galaxies that have been obtained via hundreds of hours of {\it Hubble Space Telescope} surveys over the last decade. By characterizing the multiphase structure of the CGM as a function of pressure and turbulent mach number, these simulations will provide a framework with which we can build an accurate sub-grid model for physics that will remain unresolved in large-scale simulations of galaxy formation for the foreseeable future. Last but not least, these simulations will be the first step in building a coherent theory of the thermally unstable turbulent plasmas that arise in many astrophysical systems, making our results broadly applicable to the wider astrophysics community.

%\end{flushleft}

\end{document}
