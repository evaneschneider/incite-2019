%\documentclass[12pt,letterpaper,english]{article}
\documentclass[9pt,landscape]{article}
\usepackage[T1]{fontenc} % Standard package for selecting font encodingsamely titles, headers and contents
\usepackage{txfonts} % makes spacing between characters space correctly
\usepackage{xcolor} % Driver-independent color extensions for LaTeX and pdfLaTeX.
%\usepackage[pagestyles,raggedright]{titlesec} % related with sections—n
%\usepackage{blindtext} % To create text
\usepackage{fancyhdr} % header footer placement

\usepackage[top=.7in, bottom=.7in, left=1in, right=1in] {geometry} % Margins
\usepackage{graphicx}  % Essential for adding images to you document.

\usepackage{sectsty}
\usepackage{array}
\sectionfont{\normalsize}
\subsectionfont{\normalsize}
\subsubsectionfont{\normalsize \it}

\usepackage{caption}
\captionsetup{labelsep=period}

\renewcommand{\arraystretch}{1}

\pagestyle{fancy} % allows you to use the header and footer commands

\raggedright
\begin{document}

\setlength{\parindent}{0in} % Amount of indentation at the first line of a paragraph.

\pagestyle{fancy} \lhead{The Detailed Physical Structure of the Circumgalactic Medium} \rhead{Evan Schneider} \renewcommand{%
\headrulewidth}{0.0pt}

%\colorbox{yellow} {\bf {\emph{[Refer to the guidelines for instructions in preparing the proposal. Table does not count toward}}}
%\colorbox{yellow} {\bf {\emph{[15-page project narrative count.]}}}

\textbf{Proposal Title (exactly as it appears on submission):} The Detailed Physical Structure of the Circumgalactic Medium
%
%
%Year 1 table. Please enter your data for year 1 milestones in the table below.
%
%
\vspace{-.5cm}
\begin{table}[h]
%\centering
%\caption{Table title}
%\vspace{-.15in}
\label{Tab1}
\begin{tabular}{m{4cm} m{11cm} m{2.5cm} m{3.5cm}} \\ \hline
\noalign{\smallskip}
\multicolumn{4}{l}{\textbf {Year 1}}\\ \hline
\textbf{Milestone} & \textbf{Details} & \textbf{Dates} & \textbf{Status (renewals only)} \\ \hline
\textbf{RM.A} Develop "fiducial" turbulent CGM model and run resolution study, including first petascale simulation of the CGM. & {\begin{tabular}[l]{@{}l@{}} 
\textbf {Resources:} Titan \hspace{.5cm}  \textbf{Node hours:} 0.5 M \\ 
\textbf{Filesystem storage (TB and dates):} insert text, 01/19 - 03/19 \\  
\textbf{Archival storage (TB and dates):} 03/19 - 03/20\\ 
\textbf{Software Application:} Cholla \\
\textbf{Tasks:} insert text
\textbf{Dependencies:} N/A \end{tabular}} & 01/19 - 03/19 & N/A \\ \hline
\textbf{RM.B} Run suite of moderate resolution turbulent box simulations to explore relevant parameter space. & {\begin{tabular}[l]{@{}l@{}} 
\textbf {Resources:} Titan \hspace{.5cm}  \textbf{Node hours:} 0.7M \\ 
\textbf{Filesystem storage (TB and dates):} insert text, 04/19 - 06/19 \\  
\textbf{Archival storage (TB and dates):} 06/19 - 06/20 \\ 
\textbf{Software Application:} Cholla \\
\textbf{Tasks:} 
\textbf{Dependencies:} N/A \end{tabular}} & 04/19 - 06/19 & N/A \\ \hline
\textbf{RM.C} Run additional extreme resolution models (as informed by parameter study) to produce a state-of-the-art suite of numerical simulations. & {\begin{tabular}[l]{@{}l@{}} 
\textbf {Resources:} Titan \hspace{.5cm}  \textbf{Node hours:} 1.5M \\ 
\textbf{Filesystem storage (TB and dates):} insert text, 07/19 - 09/19 \\  
\textbf{Archival storage (TB and dates):} insert text, 09/19 - 09/20 \\ 
\textbf{Software Application:} Cholla \\
\textbf{Tasks:}
\textbf{Dependencies:} RM.A, RM.B \end{tabular}} & 07/19 - 09/19 & N/A \\ \hline
\end{tabular}
\end{table}


\end{document}

