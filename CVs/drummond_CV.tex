\documentclass[11pt,letterpaper,english]{article}
\usepackage[T1]{fontenc} % Standard package for selecting font encodings
\usepackage{txfonts} % makes spacing between characters space correctly
\usepackage{xcolor} % Driver-independent color extensions for LaTeX and pdfLaTeX.
%\usepackage{blindtext} % To create text
%\usepackage{mdwlist} % mdwlist for compact enumeration/list items 
%\usepackage[pagestyles]{titlesec} % related with sections—namely titles, headers and contents
\usepackage{fancyhdr} % header footer placement

\usepackage[top=1in, bottom=1in, left=1in, right=1in] {geometry} % Margins
\usepackage{graphicx}   % Essential for adding images to you document.

\usepackage{sectsty}
\sectionfont{\large}
\subsectionfont{\normalsize}
\subsubsectionfont{\normalsize \it}

\usepackage{caption}
\captionsetup{labelsep=period}

\pagestyle{fancy} % allows you to use the header and footer commands 

\raggedright
\begin{document}

\setlength{\parindent}{0in} % Amount of indentation at the first line of a paragraph.

\pagestyle{fancy} \lhead{The Detailed Physical Structure of the Circumgalactic Medium} \rhead{Evan Schneider} \renewcommand{%
\headrulewidth}{0.0pt}



\centering {\bf Curriculum Vitae ({\emph{2-page limit}})}\\
{\bf Drummond Fielding}\\
{\bf Center for Computational Astrophysics, Flatiron Institute, drummondfielding@gmail.com} \\
{\bf Astronomy Department, University of California, Berkeley, dfielding@berkeley.edu} \smallskip

\begin{flushleft} {\bf Professional Preparation}
{\parindent 16pt

PhD, Astronomy, University of California, Berkeley, 2018 \\ 
MS, Astronomy, University of California, Berkeley, 2014 \\ 
BA, Physics \& Mathematics, Johns Hopkins University, 2012 \\ 
}

\vspace{.04in}
{\bf Appointments}
{\parindent 16pt

Flatiron Fellow, Flatiron Institute, Center for Computational Astrophysics \hfill starting Oct 2018\\ 
Graduate Student Researcher, University of California, Berkeley \hfill  2017--2018  \\
Berkeley Fellow, University of California, Berkeley     \hfill  2015--2017 \\ 
NSF Graduate Research Fellow, University of California, Berkeley  \hfill  2012--2015 \\
}

\vspace{.04in}
{\bf Five Publications Most Relevant to This Proposal}
\vspace{-6pt}
\begin{enumerate} \itemsep1pt \parskip0pt \parsep0pt
\item "The impact of star formation feedback on the circumgalactic medium", Fielding, D., et al. \textit{MNRAS}, {\bf 466}, 3810 (2017)\\ 
\item "How supernovae launch galactic winds", Fielding, D., et al. \textit{MNRAS}, {\bf 470}, L39 (2017)\\ 
\item "Numerical Simulations of Powerful Galactic Winds Driven by Clustered Supernovae", Fielding, D. \& Quataert, E. \textit{MNRAS, in press}, (2018)\\ 
\item "Supernova feedback in a local vertically stratified medium: interstellar turbulence and galactic winds", Martizzi, D.,  Fielding, D., et al. \textit{MNRAS}, {\bf 459}, 2311 (2016)\\ 
\item "Simulations of Jet Heating in Galaxy Clusters: Successes and Numerical Challenges", Martizzi, D., Quataert, E., Faucher-Giguere,C.A., Fielding, D. \textit{MNRAS, in press}, (2018)\\ 
\end{enumerate} 

\vspace{-6pt}
{\bf Research Interests and Expertise\\}
{\parindent 16pt
Co-PI Fielding's research interests are focused on understanding how the interplay of galactic winds and small scale physical processes in circumgalactic medium regulate galaxy formation. To study this process Fielding has investigated how galactic winds are launched by the collective effect of numerous supernovae, and how these winds expand into, stir, and heat the circumgalactic medium.
Fielding's work has consisted primarily of designing controlled numerical experiments to isolate specific phenomena and address well-posed questions. This work has given Fielding extensive experience using hydrodynamical simulations, and, in particular, implementing additional physical processes to couple to the hydrodynamics such as radiative cooling, thermal conduction, turbulent driving, and supernova explosions. As a graduate student Fielding served as Co-I on accepted NSF Extreme Science and Engineering Discovery Environment (XSEDE) computational proposals ``Conduction, Convection, and Thermal Instability in Hot Halos'' (TG-AST140083) and ``The Physics of Supernova Feedback: Global 3D Simulations of Galactic Disks'' (TG-AST160020) that were awarded a total of 3.6 million cpu hours. }

\vspace{.04in}
{\bf Synergistic Activities}
\vspace{-6pt}
\begin{enumerate} \itemsep1pt \parskip0pt \parsep0pt
\item Consultant / commissioning time user for the Berkeley Research Computing HPC Savio cluster
\item Public lecturer (e.g., presented on numerical astrophysics to the East Bay Astronomical Society, and taught astronomy course for 2nd and 3rd graders at North Oakland Community Charter School) \\
\item Referee for ApJ and MNRAS. \\
\end{enumerate} 

\vspace{-6pt}
{\bf Collaborators ({\emph{past 5 years including name and current institution}})} \\
{\parindent 16pt
Quataert, Eliot \hfill University of California, Berkeley \\
Faucher-Giguere, Claude-Andre\hfill Northwestern University \\
Martizzi, Davide\hfill University of California, Santa Cruz \\
McKee, Christopher\hfill University of California, Berkeley \\
Stern, Jonathan \hfill Northwestern University \\
Sharma, Prateek\hfill Indian Institute of Science \\
White, Christopher J. \hfill University of California, Santa Barabara \\
McCourt, Michael\hfill University of California, Santa Barabara \\
Klein, Richard\hfill University of California, Berkeley \\
Thompson, Todd A.\hfill The Ohio State University \\
Kim, Chang-Goo \hfill Princeton University \\
Ostriker, Eve \hfill Princeton University \\
Prochaska, J. Xavier\hfill University of California, Santa Cruz \\
Oh, Peng\hfill University of California, Santa Barabara \\
Stone, James \hfill Princeton University \\
Lecoanet, Daniel \hfill Princeton University \\
}


\end{flushleft}

\end{document}
