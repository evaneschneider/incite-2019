\documentclass[11pt,letterpaper,english]{article}
\usepackage[T1]{fontenc} % Standard package for selecting font encodings
\usepackage{txfonts} % makes spacing between characters space correctly
\usepackage{xcolor} % Driver-independent color extensions for LaTeX and pdfLaTeX.
%\usepackage{blindtext} % To create text
%\usepackage{mdwlist} % mdwlist for compact enumeration/list items 
%\usepackage[pagestyles]{titlesec} % related with sections—namely titles, headers and contents
\usepackage{fancyhdr} % header footer placement

\usepackage[top=1in, bottom=1in, left=1in, right=1in] {geometry} % Margins
\usepackage{graphicx}   % Essential for adding images to you document.

\usepackage{sectsty}
\sectionfont{\large}
\subsectionfont{\normalsize}
\subsubsectionfont{\normalsize \it}

\usepackage{caption}
\captionsetup{labelsep=period}

\pagestyle{fancy} % allows you to use the header and footer commands 

\raggedright
\begin{document}

\setlength{\parindent}{0in} % Amount of indentation at the first line of a paragraph.

\pagestyle{fancy} \lhead{The Detailed Physical Structure of the Circumgalactic Medium} \rhead{Evan Schneider} \renewcommand{%
\headrulewidth}{0.0pt}



\centering {\bf Curriculum Vitae ({\emph{2-page limit}})}\\
{\bf Drummond Fielding}\\
{\bf Astronomy Department, University of California, Berkeley, dfielding@berkeley.edu} \smallskip

\begin{flushleft} {\bf Professional Preparation}
{\parindent 16pt

PhD, Astronomy, University of California, Berkeley, 2018 \\ 
MS, Astronomy, University of California, Berkeley, 2014 \\ 
BA, Physics \& Mathematics, Johns Hopkins University, 2012 \\ 
}

\vspace{.04in}
{\bf Appointments}
{\parindent 16pt

Flatiron Fellow, Flatiron Institute, Center for Computational Astrophysics \hfill starting Oct 2018\\ 
Graduate Student Researcher, University of California, Berkeley \hfill  2017--2018  \\
Berkeley Fellow, University of California, Berkeley     \hfill  2015--2017 \\ 
NSF Graduate Research Fellow, University of California, Berkeley  \hfill  2012--2015 \\
}

\vspace{.04in}
{\bf Five Publications Most Relevant to This Proposal}
\vspace{-6pt}
\begin{enumerate} \itemsep1pt \parskip0pt \parsep0pt
\item "The impact of star formation feedback on the circumgalactic medium", Fielding, D., et al. \textit{MNRAS}, {\bf 466}, 3810 (2017)\\ 
\item "How supernovae launch galactic winds", Fielding, D., et al. \textit{MNRAS}, {\bf 470}, L39 (2017)\\ 
\item "Numerical Simulations of Powerful Galactic Winds Driven by Clustered Supernovae", Fielding, D. \& Quataert, E. \textit{MNRAS, in press}, (2018)\\ 
\item "Supernova feedback in a local vertically stratified medium: interstellar turbulence and galactic winds", Martizzi, D.,  Fielding, D., et al. \textit{MNRAS}, {\bf 459}, 2311 (2016)\\ 
\item "Simulations of Jet Heating in Galaxy Clusters: Successes and Numerical Challenges", Martizzi, D., Quataert, E., Faucher-Giguere,C.A., Fielding, D. \textit{MNRAS, in press}, (2018)\\ 
\end{enumerate} 

\vspace{-6pt}
{\bf Research Interests and Expertise}
{\parindent 16pt
CoI Fielding's research interests are focused on understanding the interplay of galactic winds and the small scale structure of the circumgalctic medium and how it serves it regulate galaxy formation. %As the primary developer of the GPU-based astrophysics code \textit{Cholla}, Schneider is an expert in hydrodynamical simulation methodology. Given the dynamic range required in cosmological simulations, many baryonic processes remain unresolved. Schneider's Ph.D. thesis and ongoing work consist of using the code \textit{Cholla} to produce petascale astrophysical simulations that reveal previously unknown details of galactic structure, including the turbulent interstellar medium, galactic outflows, and the circumgalactic medium. Schneider served as CoI of the OLCF DD Project AST107 ``Scaling the GPU-enabled Hydrodynamics Code Cholla to the Power of Titan" and DD Project AST119 ``Extending the Physics of the GPU-Enabled CHOLLA Code to the Power of Titan", and is Co-PI of the OLCF INCITE Project AST125 ``Revealing the Physics of Galactic Winds with Petascale GPU Simulations".
}

\vspace{.04in}
{\bf Synergistic Activities}
\vspace{-6pt}
\begin{enumerate} \itemsep1pt \parskip0pt \parsep0pt
%\item Primary developer and maintainer of the astrophysical hydrodynamics code, \textit{Cholla}. \\ 
%\item Member of the OLCF Users Group Executive Board. \\
%\item Presented at various University of Arizona events emphasizing the utility of HPC systems. \\ 
%\item Attended the 2017 \& 2018 OLCF User Meetings in Oak Ridge, TN. \\ 
%\item Advocate for improving the representation of minorities in the HPC community. \\
\item Referee for ApJ and MNRAS. \\
\end{enumerate} 

\vspace{-6pt}
{\bf Collaborators ({\emph{past 5 years including name and current institution}})} \\
{\parindent 16pt
Quataert, E., University of California, Berkeley \\
Faucher-Giguere, Claude-Andre, Northwestern University \\
Martizzi, D., University of California, Santa Cruz \\
Thompson, T. A., The Ohio State University \\
McCourt, M., University of California, Santa Barabara \\
Oh, P., University of California, Santa Barabara \\
}


\end{flushleft}

\end{document}
